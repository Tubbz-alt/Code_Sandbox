\chapter*{Orthorectification Notes}
\addcontentsline{toc}{chapter}{Orthorectification Notes}

This section is divided into the following sections$\cdots$

\begin{itemize}
\item[] Camera View Rectification
\item[] Perspective to Parallel Transformation
\item[] Camera Distortion Rectification
\end{itemize}
  
\section*{Camera View Rectification}
\addcontentsline{toc}{section}{Camera View Rectification}

In order to orthorectify aerial imagery, you must first correct the image camera angles
such the image appears in a ``bird's eye" view.  Here is my current approach to accomplishing this...

\begin{itemize}
\item[] Compute required parameters.
\item[] Build output image structure
\item[] Iterate over every pixel in the output image
    \begin{itemize}
    \item[] Compute the intersection of the pixel-output camera orgin line with the ground plane.
    \item[] Compute the intersection of the ground point-input camera origin with the camera's image plane.
    \item[] Compute the difference between the principle point and the image plane intersection. 
    \item[] Use this vector as the pixel location to reference the input image with. 
    \end{itemize}
\end{itemize}

\subsection*{Required Parameters}

\begin{itemize}
\item[] \gls{Camera Origin}, $P_O$
\item[] Rotation Angle, $\theta$, and the Rotation Axis $\bar{R}$.
\end{itemize}

\subsection*{Derived Parameters}

\begin{itemize}
\item[] \gls{Principle Point}, $P_P$
\item[] \gls{Principle Ground Point}, $P_G$
\end{itemize}

\subsubsection*{Computing the Principle Point}
Unless otherwise stated, the principle point on the image is assumed to be the center of the image. This may not be
the case for all cameras, however it allows for a more simple model if unknown. 

\begin{equation}
P_P = \textup{Rotation\_Matrix}(\theta, \bar{R}) \cdot \left( f \cdot ||R|| \right ) + P_O
\end{equation}

