\documentclass[10pt]{report}
\usepackage{fullpage}
\usepackage{graphicx}
\usepackage{verbatim}
\usepackage{subfig}
\usepackage{listings}

\newcommand{\HRule}{\rule{\linewidth}{0.5mm}}

\lstset{language=C++}

\begin{document}
\begin{titlepage}
\begin{center}

\vspace{10mm}

% Upper part of the page
\textsc{\LARGE Sierra Nevada Corporation}\\[1.5cm]

\vspace{80mm}

\textsc{\Large GeoImage Library}\\[0.5cm]

\vspace{10mm}

% Title
\HRule \\[0.4cm]
{ \huge \bfseries User Manual}\\[0.4cm]

\HRule \\[1.5cm]

% Author and supervisor
\begin{minipage}{0.4\textwidth}
\begin{center} \large
\emph{Author:}\\
Marvin \textsc{Smith}
\end{center}
\end{minipage}
\vfill
% Bottom of the page
{\large \today}
\end{center}
\end{titlepage}
\newpage

%**************************************************%
%                    INTRODUCTION                  %
%**************************************************%
\section*{Introduction}

The GeoImage Library is designed to allow for the loading and manipulation 
of NITF (National Imagery Transmission Format) images in OpenCV and C++ in
general.

\section*{Usage}

First, it is necessary to include the library.  This is done with a single
include statement. This proceedure is shown in Figure \ref{fig:basic01}. 


\begin{figure}[!h]
\begin{lstlisting}
#include <GeoImage>

int main( int argc, char* argv[] ){
   
   GeoImage img;
   return 0;
}

\end{lstlisting}
\caption{Process of importing the library and creating a basic image.}
\label{fig:basic01}
\end{figure}

When you compile your program, it is necessary to link the library. This
is shown in figure \ref{fig:basic02}.  Note that the build system will 
default to \texttt{/opt/local} for the path.  You may change this in the Makefile.


\begin{figure}[!h]
\begin{verbatim}
g++ hello_world.cpp -I/opt/local/include -L/opt/local/lib -lgeoimage
\end{verbatim}
\caption{Compilation example.}
\label{fig:basic02}
\end{figure}

\end{document}
