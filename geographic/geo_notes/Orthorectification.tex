\chapter*{Orthorectification Notes}
\addcontentsline{toc}{chapter}{Orthorectification Notes}

This section is divided into the following sections$\cdots$

\begin{itemize}
\item[] Camera View Rectification
\end{itemize}
  
\section*{Camera View Rectification}
\addcontentsline{toc}{section}{Camera View Rectification}

\begin{enumerate}
\item Compute the geographic extents.  AKA, find the min and max coordinates
      and construct an axis-aligned bounding box. 
\item Given the bounding box, compute the Ground Sampling Distance.  Multiply the 
      GSD with the image to determine the output image size.
\item Iterate over every pixel in the output image. 
\item For a pixel in the output image, compute its geographic world coordinate value. 
\item Compute the vector which links the pixel's world position with the camera position
      of the original image. 
\item Conduct an intelligent search of the space to look for pixels which occlude
      the original point.
\item Given the point in world coordinates, compute the pixel in the original image. 
\item Set the output image pixel. 
\end{enumerate}


\subsection*{Computing the geographic bounding box.}
\addcontentsline{toc}{subsection}{Geographic Bounding Box}

For each corner $P_{\textup{pix}}\left( X, Y\right)$, compute the world coordinate of the pixel on the surface of the 
image plane as $P_{\textup{img}}\left( X, Y\right)$.  This will enable you to compute the intersection of the point
with the surface plane as $P_{\textup{world}}$.

\begin{equation}
P_{\textup{img}} =  M_{w} M_{c} M_{p}
\end{equation}

 $M_{p}$ is the transformation which transforms the pixel to the focal plane space. The focal plane space is the
2D image surface originating around the principle point $P_o$.  This is often assumed to be the center of the image.
In addition the unit of measure is in meters and the range of the image is the length and width of the CCD surface.

\begin{equation}
M_{p} = 
\begin{bmatrix}
 \frac{X_{f}}{X_{\textup{I}}}& 0 & 0 & -\frac{X_f}{2} \\ 
0 & -\frac{Y_{f}}{Y_{\textup{I}}} & 0 & \frac{Y_f}{2} \\ 
0 & 0 & 1 &0 \\ 
0 & 0 & 0 &1 
\end{bmatrix}
\end{equation}



Where $X_{f}$ and $Y_{f}$ are the dimensions of the focal plane and $X_{\textup{I}}$ and $Y_{\textup{I}}$
are the dimensions of the input image.  It is important to note that the Y axis is scaled inversely
as image coordinates must be converted from a top-left origin to a bottom-left one.

\begin{equation}
M_{c} = 
\begin{bmatrix}
1 & 0 & 0 & 0\\ 
0 & 1 & 0 & 0\\ 
0 & 0 & -f & 0\\ 
0 & 0 & 0 & 1
\end{bmatrix}
\end{equation}


$M_{c}$ converts the pixel space into the camera coordinate system. For convenience, this is merely a 3D 
coordinate system identical to the focal plane space, where the z axis is the negative focal length.  This
is important as this will be identical to the world coordinate system except that the camera is rotated
into position and translated by the world coordinates of the camera. 

\begin{equation}
M_{w} = \begin{bmatrix}
1 & 0 & 0 & T_{Cx}\\ 
0 & 1 & 0 & T_{Cy}\\ 
0 & 0 & 1 & T_{Cz}\\ 
0 & 0 & 0 & 1
\end{bmatrix} 
\mathbf{M_{\textup{Quaternion}}}
\end{equation}

Where $T_{C}$ is the position of the camera in the world coordinate system and
$M_{Quaternion}$ is the matrix derived from the quaternion. 

This will give you the 3D coordinate of the pixel on the focal plane.  You still need
to translate this into the actual location on the surface.  In order to compute this, you
need to compute the intersection between the ray connecting the camera origin and the 
point on the focal plane with the surface of the Earth.

\begin{equation}
\mu = \frac{N \cdot \left(P_n - P_c\right) }{N \cdot \left(P_p - P_c\right)}
\end{equation}

\begin{equation}
P_{\textup{gnd}} = P_c - \mu \cdot \left( P_{p} - P_{c} \right)
\end{equation}

Where $\mu$ is the scale factor of the ray, $P_n$ is the nadir point, $P_c$ is the camera origin, 
$P_p$ is the principle point, and $N$ is the earth normal pointing $\left( 0, 0, 1\right)$.
 
 
Applying these equations to each of the 4 corners will determine the min and max values for the 
coordinate system. 

\subsection*{Computing the GSD}
\addcontentsline{toc}{subsection}{Computing the GSD}

The ground sampling distance is the distance per pixels, measured in this project as \emph{meters per pixel}.
This is computed using the bounding box of the image computed in the previous section.  Incorporating the 
rotation into this would help, however I currently am not sure how to address this. 

I currently compute the GSD for the x axis as the average between the span of the top row and bottom row. 
This span is then divided by the number of pixels.  For example, if top row of a 1000 pixel wide image is 5000 meters wide and the bottom
row is 1000 meters (the image is rotated), then the GSD is (5000 + 1000)/1000 = 3 meters per pixel.  The 
GSD for the Y is the same except column heights. The final GSD is the smaller of the two.  A smaller GSD is safer as you can 
scale the image down and not lose data. 

 
\subsection*{Iterate over image}
\addcontentsline{toc}{subsection}{Iterate over image}

Once the parameters of the output image are determined, it is necessary to iterate every pixel and find the relationship between its position in the input image. 
Since we create a bounding box with is axis-aligned, we can easily find the ground position of the pixel in the output image. 

\begin{equation}
P_{\textup{plane}} = \begin{bmatrix}
\frac{\textup{bbox.max}.x}{I_{cols}} & 0 & \textup{bbox.min}.x\\ 
 0 & \frac{\textup{bbox.max}.y}{I_{rows}} & \textup{bbox.min}.y\\ 
  0 & 0 & 1 
  \end{bmatrix}
  \cdot
  \begin{bmatrix}
  X \\
  Y \\
  1
  \end{bmatrix}
\end{equation}

Where $\textup{bbox}$ is the geographic bounding box of the image, $I$ is the output image, and $X,Y$ are the image coordinates you are iterating. 

From this, compute the plane-line intersection with the image-plane world coordinate and the ground.  Use the camera position as the other point. 

This equation will yield the ground coordinates of the output image pixel. 

\subsection*{Geographic Coordinate to Image Pixel}
\addcontentsline{toc}{subsection}{Geographic Coordinate to Image Pixel}

Now that we know the geographic coordinate of the output pixel, we need to compute the input image coordinate.  This will allow us to copy its pixel to the output image. 
Note that this current technique does zero interpolation, so this should be eventually implemented. 

\begin{itemize}
\item Compute a line-plane intersection with the line consisting of the camera origin and the ground coordinate.  The plane will conist of the camera plane normal and
      the principle point. This is the point in the world where the camera-ground line intersects the input camera's focal plane. 
\item Convert the world coordinates into camera coordinates by subtracting out the camera principle point world coordinate.  Then rotate to normalize the angle. 
\item Convert the camera coordinates into 2D image plane coordinates.  Multiply the point by the focal plane size, and subtract half. This will bring the point to [0,1]
      range on the focal plane.  Then multiply by the image size to get the real coordinates in pixels. 
\end{itemize}

\begin{equation}
P_{\textup{pix}} = 
\begin{bmatrix} 
\textup{img}_\textup{cols} & 0 & 0 & 0 \\
0 & \textup{img}_\textup{rows} & 0 & 0 \\
0 & 0 & 1 & 0 \\
0 & 0 & 0 & 1 
\end{bmatrix}
\cdot
\begin{bmatrix}
1 & 0 & 0 & -\textup{F}_x\\ 
0 & 1 & 0 & \textup{F}_y\\ 
0 & 0 & 1 & 0\\ 
0 & 0 & 0 & 1
\end{bmatrix}
\cdot
R^{-1}_{\textup{quat}} \cdot \begin{bmatrix}
1 & 0 & 0 & -P_{x_\textup{prin}} \\ 
0 & 1 & 0 & -P_{y_\textup{prin}}\\ 
0 & 0 & 1 & -P_{z_\textup{prin}}\\ 
0 & 0 & 0 & 1
\end{bmatrix} \cdot P_{\textup{gnd}} 
\end{equation}

Where $F$ is the dimensions of the focal plane, $\textup{img}$ are the dimensions of the input image, and $P_\textup{prin}$ is the principle point.


\subsection*{Digital Elevation Model Correction}
\addcontentsline{toc}{subsection}{Digital Elevation Model Correction}

In order to fix the occlusions caused by the perspective view, it is necessary to search and 
compute all points on the dem which intersect the line from the camera origin and the world
point.



