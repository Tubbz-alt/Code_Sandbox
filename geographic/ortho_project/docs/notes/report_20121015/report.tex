\documentclass[12pt]{report}

% Libraries
\usepackage{graphicx}
\usepackage{subfig}
\usepackage{fullpage}

\title{Ortho-Project Notes}
\author{Marvin Smith}
\date{\today}

\begin{document}

\maketitle

\section*{Current Progress}

\begin{itemize}
\item Revamped and restructured the entire ortho module. Reviewed the math and structured the code in an 
      effort to improve readability. 
      \begin{itemize}
      \item Note: DEM Correction is still being worked on.
      \end{itemize}
\item Built baseline structure for OrthoProject unit tests.  Need to start validating that the large number
of geometric utilities are accurate to a desired level of precision.  I have tested them during development, however this is something that 
needs to be implemented long-term. 
\item Beginning to implement coordinate conversion utilities, testing DEM Module, and working on utilities for 
parsing GS1 NITF Headers.  This is required to integrate DTED and geographic imagery as we are dealing with 
complex coordinate systems and different units of measure.
\end{itemize}

\section*{Metrics}

\subsection*{GeoImage}
\begin{itemize}
\item Lines of Code: 10223
\item Libraries: OpenCV, Boost Filesystem, GDAL
\item New Features since 9/14
    \begin{itemize}
    \item \texttt{GEO::GeoImage::get\_image( int CV\_TYPE )}
        \begin{itemize}
        \item You can now retrieve the image from a GeoImage in any format you wish.
        \end{itemize}

    \item \texttt{GEO::GS2::GS2\_Header}
        \begin{itemize}
        \item Get and set functions for various GS2 NITF header items.
        \item Currently aircraft tail number and focal length. More planned.
        \end{itemize}
    \item \texttt{GEO::CoordinateBase}
        \begin{itemize}
        \item Framework in place to implement Coordinate Conversion using GDAL.
        \item Currently containers in place for UTM, Geodetic, and LambertConic
        \item Still developing overall architecture for API
        \end{itemize}
    \end{itemize}
\end{itemize}

\subsection*{OrthoProject}
\begin{itemize}
\item Lines of Code: 5333
\item Libraries: OpenCV, Boost Filesystem, GeoImage
\end{itemize}

\section*{Issues}

\subsection*{Pixel-by-pixel into Homorgraphy}
I don't see a means of converting the pixel-by-pixel search into a homography.  This is because
once we have the ground point, we must search for occlusions.  A homography has no method of testing 
for occlusions as it cannot search. I think the best method is to create a highly optimized search function, 
then apply each pixel independently as a thread in a mass parallel pipeline. 

\subsection*{Overall Scope}
Integration of DTED is going to require a lot more cycles than I anticipated.  This project
requires well-developed algorithms and libraries to manage the large number of coordinate systems, file types, 
header requirements, and processor/memory demands.  DTED in particular is dangerous as loading large tiles can 
kill system memory.  Once I get a baseline working for small samples, I need to investigate a more robust and large scale method 
for handling very large datasets. Perhaps a database structure, oct-tree, or other conceptual design will work better. Also I have not begun
to consider what happens if holes are not filled, as the value is $-2^{16}$. This along with the fact that most geographic coordinates
do not fall directly along a dted post, interpolation between coordinate posts will further tax our algorithm.  

\section*{Future Plans}

\begin{itemize}
\item Finish OrthoProject code, specifically...
    \begin{itemize}
    \item Make DEM Correction work for test imagery 
    \item Make DEM Correction work for Geographic imagery
    \item Read camera intrinsics for Geographic imagery from nitf headers.
    \item Build test bench for script-level testing.
    \item Create static library build to run on better machines (GSMC Server have 64 cores).
    \item Begin laying down framework for parallelization.
    \end{itemize}
\item Create tools to determine camera intrinsics from geographic imagery.
    \begin{itemize}
    \item Have user select N+ ground points
    \item Create system of equations required to solve
    \item Use least squares if solving homography
    \item If solveable equations, possibly Levenburg Marquardt
    \item Finally, I have very well tested Genetic Algorithm code.
    \item Create a single file format for saving camera models. 
    \end{itemize}
\item Once all of this is solved, begin stitching/bundle adjustment process
\end{itemize}

\end{document}

